\setcounter{page}{2}
Ein zentraler Begriff in der Differentialgeometrie/-topologie ist der der Mannigfaltigkeit. Das wohl einfachste nicht triviale Beispiel ist dabei gerade die \(1\)-Sph\"are, also die Menge \(\left\{(x,y)\in\mathbb{R}^2\colon x^2+y^2=1\right\}\). Intuitiv erscheint es plausibel, dass diese in irgendeiner Art \"ahnlich zu der reellen Gerade ist. Idealerweise (im Sinne der Topologie) lie\ss e sich also Hom\"oomorphismus \(\phi\colon\mathbb{R}\to S^1\) angeben. Da jedoch mit Hilfe einer stereographischen Projektion \(\mathbb{R}\cong S^1\setminus\{e_1\}\) folgt, kann solch ein \(\phi\) nicht existieren. Dies liegt anschaulich daran, dass ein Hom\"oomorphismus die reelle Gerade an den \glqq Enden\grqq{} nicht \glqq zusammenkleben\grqq{} kann. Andererseits k\"onnen wir \(S^1\) mit offenen Mengen \"uberdecken, die hom\"oomorph zu dem euklidischen Raumes sind. Dies w\"aren zum Beispiel \(S^1\setminus\{p_1\}\cong\mathbb{R}\) und \(S^1\setminus\{p_2\}\cong\mathbb{R}\) mit \(p_1\not=p_2\), erneut mit zwei stereographische Projektionen. Nun begibt es sich, dass Gleichungen in der Mathematik recht h\"aufig auftreten, und das soeben beschriebene Ph\"anomen keinen Einzelfall darstellt. Ebenso sind auch andere R\"aume, die sich lokal wie der euklidische Raum verhalten, nicht selten anzutreffen. Solche Konstrukte werden auch (topologische) Mannigfaltigkeiten genannt. Insbesondere ist es oft recht leicht eine Mannigfaltigkeit abstrakt zu definieren, anstatt sie direkt als Teilmenge des \(\mathbb{R}^n\) aufzufassen und explizite Karten anzugeben, obwohl bekannterweise eine solche Einbettung existiert. Ein bekanntes Beispiel hierf\"ur ist der Torus, der sich simpel als \(S^1\times S^1\) definieren l\"asst. Als Produkt von Mannigfaltigkeiten ist \(\mathbb{T}\) dann erneut eine Mannigfaltigkeit, und sind \(V_1\cong U_1\subset S^1\) und \(V_2\cong U_2\subset S^1\) eine gew\"unschte \"Uberdeckung von \(S^1\), so sind die \(U_i\times U_j\cong V_i\times V_j\) f\"ur \(1\leq i,j\leq2\) eine \"Uberdeckung von \(\mathbb{T}\). Den Torus als Teilmenge des dreidimensionalen Raumes aufzufassen ist im Vergleich aufw\"andiger. Ebenso ist es von Interesse Konzepte wie die Differenzierbarkeit auf Mannigfaltigkeiten zu untersuchen, was zu zus\"atzlichen Strukturen f\"uhrt. Der Kernbegriff der folgenden Arbeit ist hierbei die \textit{differenzierbare Mannigfaltigkeit}, den wir sp\"ater noch pr\"azisieren wollen. Wie bereits am Beispiel des Torus erw\"ahnt, ist es unter Umst\"anden m\"oglich, eine Mannigfaltigkeit in den euklidischen Raum einzubetten. Dies ist besonders hilfreich, da auf reellen Mannigfaltigkeiten einige Konzepte wie zum Beispiel die Integration bereits definiert sind, was auf beliebigen R\"aumen nicht immer der Fall sein wird. Wir wollen im Folgenden den schwachen Whitneyschen Einbettungssatz beweisen, welcher uns garantiert, dass eine differenzierbare \(m\)-dimensionale Mannigfaltigkeit stets eine Einbettung in den \(\mathbb{R}^{2m+1}\) besitzt. Wie der Name bereits impliziert, existiert ebenso ein starken Einbettungssatz, der eine Einbettung in den \(\mathbb{R}^{2m}\) garantiert. Dieser wurde 1944 von Hassler Whitney bewiesen \cite{whitney1944intersect} und baut auf dem schwachen Einbettungssatz auf, ben\"otigt jedoch noch einige etwas aufw\"andigere Konzepte. Nun zeigen Beispiele wie die eindimensionale Mannigfaltigkeit \(S^1\), die sich nicht in den \(\mathbb{R}^1\) einbetten l\"asst, dass sich dieses Ergebnis im Allgemeinen nicht verbessern l\"asst. Andererseits l\"asst sich jede offene Teilmenge \(U\subseteq\mathbb{R}^m\) (die sich ebenso als \(m\)-dimensionale Mannigfaltigkeit auffassen l\"asst) bereits trivial in den \(\mathbb{R}^m\) einbetten, in Sonderf\"allen ist also auch eine Einbettung in einen kleineren euklidischen Raum m\"oglich.

Die folgende Beweiskette basiert im Kern auf \cite{broecker1973differentialtopologie}, wobei einige S\"atze zur Parakompaktheit auf \cite{persson2014embedding} basieren. 